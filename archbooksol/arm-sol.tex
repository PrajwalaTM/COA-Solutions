
\vskip 1cm
\section*{Exercises}
\vskip 1cm

\setcounter{Exercise}{0}
\setcounter{Answer}{0}


\section*{Basic ARM Instructions}

\begin{ExerciseList}

\Exercise
Translate the following code in C to the
ARM instruction set using a minimum number of instructions.
Assume the variables $a$, $b$, $c$, $d$ and $e$ are
32-bit integers and stored in $r0$, $r1$, $r2$, $r3$ and $r4$ respectively. 

\begin{enumerate}
	\item[(a) ] 
\begin{Verbatim}
a=a+b+c+d+e;
\end{Verbatim}
\item[(b) ] 
\begin{Verbatim}
a=b+c;
d=a+b;
\end{Verbatim}
	\item[(c) ]
\begin{Verbatim}
a=b+c+d;
a=a+a;
\end{Verbatim}
	\item[(d) ] 
	\begin{Verbatim}
a=2*a+b+c+d;
\end{Verbatim}
	\item[(e) ] 
\begin{Verbatim}
a=b+c+d; 
a=3*a;
\end{Verbatim}

\end{enumerate}

\Answer:
\begin{enumerate}
	\item[(a) ]
\begin{Verbatim}[frame=single]
add r0, r0, r1
add r0, r0, r2
add r0, r0, r3
add r0, r0, r4
\end{Verbatim}
	\item[(b) ]
\begin{Verbatim}[frame=single]
add r0, r1, r2
add r3, r0, r1
\end{Verbatim}
	\item[(c) ]
\begin{Verbatim}[frame=single]
add r0, r1, r2
add r0, r0, r3
add r0, r0, r0
\end{Verbatim}
	\item[(d) ]
\begin{Verbatim}[frame=single]
add r0, r1, r0, lsl #1
add r0, r0, r2
add r0, r0, r3
\end{Verbatim}
	\item[(e) ]
\begin{Verbatim}[frame=single]
add r0, r1, r2
add r0, r0, r3
add r0, r0, r0, lsl #1
\end{Verbatim}
\end{enumerate} 


\Exercise
Translate the following pieces of code from the 
ARM assembly language to a high level language.
Assume that the variables $a$, $b$, $c$, $d$ and $e$ (containing integers)
are stored in the registers $r0$, $r1$, $r2$, $r3$ and $r4$ respectively.

\begin{enumerate}
\item[(a) ] 
\begin{Verbatim}[frame=single]
add r0, r0, r1
add r0, r0, r2
add r0, r0, r3
\end{Verbatim}
\item[(b) ]
\begin{Verbatim}[frame=single]
orr r0, r0, r1, lsl #1
and r1, r0, r1, lsr #1
\end{Verbatim}
\item[(c) ]
\begin{Verbatim}[frame=single]
add r0, r1, r2
rsb r1, r0, r2
\end{Verbatim}
\item[(d) ]
\begin{Verbatim}[frame=single]
add r0, r1, r2
add r0, r3, r4
add r0, r0, r1
\end{Verbatim}
\item[(e) ]
\begin{Verbatim}[frame=single]
mov r0 #1, lsl #3
mov r0, r0, lsr #1
\end{Verbatim}
\end{enumerate}


\Answer:
\begin{enumerate}
\item[(a) ]
$a=a+b+c+d$;
\item[(b) ]
$a=a\:|\:(b*2)$;\\
$b=a\:\&\:(b/2)$;
\item[(c) ]
$a=b+c$; \\
$b=c-a$;
\item[(d) ]
$a=d+e+b$;
\item[(e) ]
$a=4$;
\end{enumerate}



\Exercise Answer the following:
\begin{enumerate}
\item[(a) ] Write the smallest possible ARM assembly program to load the constant 0xEFFFFFF2 into register
r0.
\item[(*b) ] Write the smallest possible ARM assembly program to load the constant 0xFFFD67FF into register
r0.
\end{enumerate}

\Answer:
\begin{enumerate}
\item[(a) ]
\begin{Verbatim}[frame=single]
mvn r0, #0xD
\end{Verbatim}
\item[(b) ]
\begin{Verbatim}[frame=single]
mvn r0, #0x29800
\end{Verbatim}
\end{enumerate}

\Exercise[difficulty=1]
Using valid ARM assembly instructions, load the constant, 
0xFE0D9FFF, into register r0. Try do to it in the minimum number of instructions. DO NOT use pseudo-instructions or assembler directives.

\Answer 0xFE0D9FFF = $(1111\: 1110\: 0000\: 1101\: 1001\: 1111\: 1111\: 1111)_2$
\begin{Verbatim}[frame=single]
mvn r0, 0x26000
sub r0, r0, 0x1F00000
\end{Verbatim}


\Exercise
Can you give a generic set of ARM instructions or a methodology using which you can load any 32 bit immediate value into a register?
Try to minimise the number of instructions. %Also show that the bound on the number of instructions is tight.

\Answer
Arm architecture allows 12 bits for specifying an immediate with 8 bits of payload and 4 bits to specify rotation. So a 32-bit number can be broken down into 4 sets of 8-bit numbers and then loaded using add instructions.
\\
Example: \\Suppose we want to load the immediate 0xAAAAAAAA into R0. Following are the steps:
\begin{Verbatim}[frame=single]
mov r0, #0xAA
add r0, r0, #0xAA00
add r0, r0, #0xAA0000
add r0, r0, #0xAA000000
\end{Verbatim}



\Exercise
Convert the following C program to ARM assembly. 
Store the integer, $i$, in register $r0$. Assume that the
starting address of array $a$ is saved in register $r1$, and the starting address of array $b$ is saved in register
$r2$.

\begin{Verbatim}[frame=single]
int i;
int b[500];
int a[500];
for(i=0; i < 500; i++) {
    b[i] = a[a[i]];
}
\end{Verbatim}

\Answer:
\begin{Verbatim}[frame=single]
l1:
	ldr r3, [r1, #0]
	add r1, r1, #4
	ldr r4, [r3, #0]
	str r4, [r2, #0]
	add r2, r2, #4
	add r0, r0, #1
	cmp r0, #500
	bne l1
	mov pc, lr
main:
	mov r0, #0
	bl l1
\end{Verbatim}

\Exercise[difficulty=2]
Consider the instruction, {\em mov lr, pc}. Why does this instruction add 8 to the PC, and use that value to set
the value of $lr$? When is this behaviour helpful?
\end{ExerciseList}

\section*{Assembly Language Programming}

\begin{itemize}
\item For all the questions below, assume that two specialised functions, $\_\_div$ and $\_\_mod$, are available.
The $\_\_div$ function divides the contents of $r1$ by the contents of $r2$, and saves the result in $r0$.
Similarly, the $\_\_mod$ function is used to divide $r1$ by $r2$, and save
the remainder in $r0$. Note that in this case both the functions
perform integer division.
\end{itemize}

\vskip 2cm

\begin{ExerciseList}

\Exercise
Write an ARM assembly language program to compute the
2's complement of a number stored in $r0$.
\Answer
Code:
\begin{Verbatim}[frame=single]
rsb r1, r0, #0
\end{Verbatim}



\Exercise
Write an ARM assembly language program that subtracts two 
64-bit integers stored in four registers.

Assumptions:
\begin{itemize}
\item Assume that you are subtracting $A - B$
\item $A$ is stored in register, $r4$ and $r5$. The MSB is in $r4$, and the LSB is in $r5$.
\item $B$ is stored in register, $r6$ and $r7$. The MSB is in $r6$, and the LSB is in $r7$.
\item Place the final result in $r8$(MSB), and $r9$(LSB). 
\end{itemize}

\Answer :
\begin{Verbatim}[frame=single]
sub r9, r5, r7
cmp r5, r7
sub r8, r4, r6
sublo r8, r8, #1
\end{Verbatim}
An alternate smaller code:
\begin{Verbatim}[frame=single]
subs r9, r5, r7
sbc r8, r4, r6
\end{Verbatim}


\Exercise
Write an assembly program to add two 96-bit numbers $A$
 and $B$ using the minimum number of instructions.
$A$ is stored in three registers $r2$, $r3$ and $r4$ 
with the MSB in $r2$ and LSB in $r4$.
$B$ is stored in registers $r5$, $r6$ and $r7$ with the
 MSB in $r5$ and LSB in $r7$. Place the final result in $r8$(MSB), $r9$ and $r10$(LSB).

\Answer :
\begin{Verbatim}[frame=single]
adds r10, r4, r7
adcs r9, r3, r6
adc r8, r2, r5
\end{Verbatim}

\Exercise
Write an ARM assembly instruction code to count the number of 1's in a 32 bit number.

\Answer :
\begin{Verbatim}[frame=single]
l0:
	and r1, r0, r3
	mov r0, r0, lsl #1
	cmp r1, #0
	addne r4, r4, #1
	cmp r0, #0
	bne l0
	mov pc, lr
count:
	mov r4, #0
	mov r0, #1
	bl l0
\end{Verbatim}


\Exercise
Given a 32-bit integer in $r3$,
write an ARM assembly program to count the number of 1 to 0 transitions in it. 

\Answer Code:
\begin{Verbatim}[frame=single]
l1:
	and r1, r3, r0
	mov r0, r0, lsl #1
	and r2, r3, r0
	sub r2, r2, r1, lsl #1
	cmp r2, r0
	addeq r4, r4, #1
	cmp r0, #0x80000000
	bne l1
	mov pc, lr
transition:
	mov r4, #0
	mov r0, #1
	bl l1
\end{Verbatim}


\Exercise[difficulty=1]
Write an ARM assembly program that checks if a 32-bit number is a palindrome.
Assume that the input is available in $r3$. The program should set $r4$
 to 1 if it is a palindrome, otherwise $r4$ should have 0.
A palindrome is a number which is the same when read from both
 sides. For example, 1001 is a 4 bit palindrome.

\Answer Code:
\begin{Verbatim}[frame=single]
l1:
	and r1, r3, r0
	mov r0, r0, lsl #1
	cmp r1, #0
	mov r1, r1, lsl #1
	mov r2, r2, lsl #1
	addne r2, r2, #1
	cmp r0, #0
	bne l1
	mov pc, lr
palin:
	mov r4, #0
	mov r0, #1
	mov r2, #0
	bl l1	
	cmp r3, r2
	moveq r4, #1
\end{Verbatim}


\Exercise
Design an ARM Assembly Language program that will examine a 32-bit 
value stored in $r1$ and count the number of contiguous sequences of 1s. For example, the value:
\begin{displaymath}
0111 0001 0001 1110 1100 0111 0001 1111 
\end{displaymath}
contains six sequences of 1s. Write the final value in register r2. Use conditional instructions
as much as possible.

\Answer Following is the program
\begin{Verbatim}[frame=single]
l0:
	cmp r4, #0
	moveq r4, #1
	addeq r2, r2, #1
	mov pc, lr
l1:
	and r3, r1, r0
	cmp r3, #0
	stmfd sp!, {lr}
	blne l0
	ldmfd sp!, {lr}
	cmp r3, #0
	moveq r4, #0
	mov r0, r0, lsl #1
	cmp r0, #0
	bne l1
	mov pc, lr
count:
	mov r2, #0
	mov r0, #1
	mov r4, #0
	bl l1
\end{Verbatim}


\Exercise[difficulty=2]
In some cases, we can rotate an integer to the right by $n$  positions (less than or equal to 31) so that we obtain the same number.
For example: a 8-bit number 11011011 can be right rotated by 3 or 6
places to obtain the same number.
Write an ARM assembly program
to {\em efficiently} count the number of ways we can rotate a number to the right such that the result is 
equal to the original number. 
\Answer:
\begin{Verbatim}[frame=single]
l0:
	mov r1, #32
	div r4, r1, r0
	mod r2, r1, r0
	cmp r2, #0
	subeq r4, r4, #1
	mov pc, lr
l1:
	cmp r1, r3
	mov r1, r1, ror #1
	add r0, r0, #1
	bne l1
	stmfd sp!, {lr}
	bl l0
	ldmfd sp!, {lr}
	mov pc, lr
count:
	mov r4, #0
	mov r0, #1
	mov r1, r3, ror #1
	bl l1
\end{Verbatim}

\Exercise
Write an ARM assembly program to load and store an integer from memory, where the memory saves it in the
big endian format.

\Answer
Load instruction:
\begin{Verbatim}[frame=single]
ldrb r2, [r0, #0]
mov r1, r2
ldrb r2, [r0, #-1]
add r1, r1, r2, lsl #8
ldrb r2, [r0, #-2]
add r1, r1, r2, lsl #16
ldrb r2, [r0, #-3]
add r1, r1, r2, lsl #24
\end{Verbatim}

Store instruction:
\begin{Verbatim}[frame=single]
and r2, r1, #0xFF
strb r2, [r0, #0]
and r2, r1, #0xFF00
mov r2, r2, lsr #8
strb r2, [r0, #-1]
and r2, r1, #0xFF0000
mov r2, r2, lsr #16
strb r2, [r0, #-2]
and r2, r1, #0xFF000000
mov r2, r2, lsr #24
strb r2, [r0, #-3]
\end{Verbatim}






\Exercise
Write an ARM assembly program to find out if a number is prime using a recursive algorithm.

\Answer:
\begin{Verbatim}[frame=single]
l1:
	mod r5, r3, r4
	cmp r5, #0
	addeq r2, r2, #1
	add r4, r4, #1
	cmp r4, r6
	ble l1
	mov pc, lr
prime:
	mov r7, #0
	mov r3, r0
	mov r6, r0, lsr #1
	mov r4, #1
	mov r2, #0
	bl l1
	cmp r2, #1
	moveq r7, #1
\end{Verbatim}




\Exercise[difficulty=1]
Suppose you decide to take your ARM device to some place with a high amount of radiation,
which can cause some bits to flip, and consequently corrupt data. 
Hence, you decide to store a single bit checksum, which stores the parity of all the other bits, 
at the LSB position of the number (Essentially you can now store only 31 bits of data in a register).
Write an ARM assembly program, which adds the two numbers taking care of the checksum. 
Assume that no bits flip while the program is running. 

\Answer Code
\begin{Verbatim}[frame=single]
l1:
	and r2, r5, r0
	cmp r2, #0
	eorne r1, r1, #1
	mov r0, r0, lsl #1
	cmp r0, #0
	bne l1
	add r5, r5, r1
	mov pc, lr
add:
	mov r0, #2
	mov r1, #0
	mvn r2, #1
	and r3, r3, r2
	and r4, r4, r2
	add r5, r3, r4
	bl l1
\end{Verbatim}





\Exercise[difficulty=1]
Let us encode a 32 bit number by using 2 bits to represent 1 bit. We shall represent logical 0 by 
01, and logical 1 by 10. 
Now let us assume that a 16 bit number 
is encoded and stored in a 32 bit register $r3$. 
Write a program in ARM assembly to convert it back into a 16 bit number, and save the result in $r4$. 
Note that 00 and 11 are invalid inputs and
indicate an error. The program should set $r5$ to 1 in case of an error; otherwise, $r5$ should be 0.

\Answer Program:
\begin{Verbatim}[frame=single]
l1:
	and r1, r3, r0
	mov r4, r4, lsl #1
	cmp r1, r9
	movlo r5, #1
	addeq r4, r4, #1
	cmp r1, r8
	movhi r5, #1
	mov r0, r0, lsr #2
	mov r8, r8, lsr #2
	mov r9, r9, lsr #2
	cmp r0, #0
	bne l1
	mov pc, lr
decode:
	mov r5, #0
	mov r4, #0
	mov r0, #0xC0000000
	mov r8, #0x80000000
	mov r9, #0x40000000
	bl l1
\end{Verbatim}



\Exercise[difficulty=2]
Write an ARM assembly program to convert a 32-bit number to its 12 bit immediate form, 
if possible, with first 4 bits for rotation and next 8 bits for the payload. 
If the conversion is possible, set $r4$ to 1 and store the result in $r5$, 
otherwise, $r4$ should be set to 0. Assume that the input number is available in register $r3$.

\Answer The code:
\begin{Verbatim}[frame=single]
l0:
	and r2, r3, r1
	mov r8, #0
	cmp r2, #0
	movne r8, #1
	cmp r6, #16
	andlt r8, r8, #1
	moveq r4, #0
	moveq pc, lr
	add r6, r6, #1
	cmp r8, #1
	moveq r3, r3, ror #2
	beq l0	
	cmp r2, #0
	movne r4, #0
	addeq r5, r3, r6, lsl #8
	mov pc, lr
immediate:
	mov r4, #1
	mvn r1, #0xFF
	mvn r6, #0
	mov r7, #0
	bl l0
\end{Verbatim}





\Exercise[difficulty=2]
Suppose you are given a 32-bit binary number. You are told that the number has exactly one bit equal to 1;
the rest of the bits are 0. Provide a fast algorithm to find the location of that bit. 
Implement the algorithm in ARM assembly. Assume the input to be available in $r9$. 
Store the result in $r10$.

\Answer
We will use a technique similar to binary search. To check if bit 1 is present to left or right of the middle pivot position, we will start with a filter, which is a sequence of continuous 1's starting from MSB up to the pivot. If bit 1 is left of pivot, new pivot is calculated (as in binary search) and filter is modified using LSL instruction. If bit 1 is right of pivot, new pivot is again calculated and the filter is modified using ASR instruction. Following is the code to implement it.
\begin{Verbatim}[frame=single]
@left = r0
@right = r1
@middle = r2
@filter = r3
l0: 
	@1 to left of middle
	add r1, r2, #1
	add r5, r0, r1
	mov r5, r5, lsr #1
	sub r7, r5, r2
	mov r3, r3, lsl r7
	mov r4, r4, lsl r7
	mov r2, r5
	mov pc, lr
l1:
	@1 to right of middle
	sub r0, r2, #1
	add r5, r0, r1
	mov r5, r5, lsr #1
	sub r7, r2, r5
	mov r3, r3, asr r7
	mov r4, r4, lsr r7
	mov r2, r5
	mov pc, lr
control:
	and r8, r9, r4
	cmp r8, #0
	movne r10, r2
	movne pc, lr
	and r8, r9, r3
	cmp r8, #0
	stmfd sp!, {lr}
	blne l0
	ldmfd sp!, {lr}
	cmp r8, #0
	stmfd sp!, {lr}
	bleq l1
	ldmfd sp!, {lr}
	b control
search:
	mov r0, #31
	mov r1, #0
	mov r2, #15
	mov r3, #0xff000000
	add r3, r3, #0xff0000
	mov r4, #0x10000
	bl control
\end{Verbatim}




\Exercise[difficulty=3]
Write an ARM assembly language program to find the greatest common divisor of two binary numbers $u$ and $v$.
Assume the two inputs (positive integers) 
to be available in $r3$ and $r4$. Store the result in $r5$.
[HINT: The gcd of two even numbers $u$ and $v$ is $2*gcd(u/2,v/2)$]

\Answer
The algorithm, also known as Stein's algorithm is as follows:

\begin{enumerate}
	\item[1. ] $gcd (v, v) = v$
	\item[2. ] $gcd(0, v) = v$, and $gcd(u, 0) = u$ 
	\item[3. ] If $u$ and $v$ are both even, then $gcd(u, v) = 2\:.\:gcd(u/2, v/2)$
	\item[4. ] If $u$ is even and $v$ is odd, then $gcd(u, v) = gcd(u/2, v)$, because $2$ is not a common divisor. Similarly for the other case, then $gcd(u, v) = gcd(u, v/2)$.
	\item[5. ] If $u$ and $v$ are both odd:
	\begin{enumerate}
		\item[i) ] If $u \geq v$, then $gcd(u, v) = gcd((u - v)/2, v)$ 
		\item[ii) ] If $u < v$, then $gcd(u, v) = gcd((v - u)/2, u)$
	\end{enumerate}
\end{enumerate}
Following is the code in ARM assembly language:
\begin{Verbatim}[frame=single]
l0:
	moveq r3, r3, lsr #1
	moveq r4, r4, lsr #1
	add r6, r6, #1
	mov pc, lr
l1:
	cmp r2, #1
	moveq r3, r3, lsr #1
	movne r4, r4, lsr #1
	mov pc, lr
l2:
	cmp r3, r4
	subge r3, r3, r4
	movge r3, r3, lsr #1
	movlt r7, r3
	sublt r3, r4, r3
	movlt r3, r3, lsr #1
	movlt r4, r7
	mov pc, lr
bcase:
	mov r0, #0
	cmp r3, r4
	moveq r5, r3
	addne r0, r0, #1
	cmp r3, #0
	moveq r5, r4
	addne r0, r0, #1
	cmp r4, #0
	moveq r5, r3
	addne r0, r0, #1
	mov pc, lr
case1:
	and r1, r3, #1
	and r2, r4, #1
	add r1, r1, r2
	cmp r1, #0
	stmfd sp!, {lr}
	bleq l0
	ldmfd sp!, {lr}
	cmp r1, #1
	stmfd sp!, {lr}
	bleq l1
	ldmfd sp!, {lr}
	cmp r1, #2
	stmfd sp!, {lr}
	bleq l2
	ldmfd sp!, {lr}
	mov pc, lr
control:
	stmfd sp!, {lr}
	bl bcase
	ldmfd sp!, {lr}
	cmp r0, #3
	moveq r3, r3, lsl r6
	moveq pc, lr
	stmfd sp!, {lr}
	bl case1
	ldmfd sp!, {lr}
	b control
gcd:
	mov r6, #0
	bl control
\end{Verbatim}

\end{ExerciseList}

\subsection*{ARM Instruction Encoding}

\begin{ExerciseList}
\Exercise
How are immediate values encoded in the ARM ISA?

\Exercise
Encode the following ARM instructions. Find the opcodes for 
instructions from the ARM architecture manual~\cite{armbook}.

\begin{enumerate}[i) ]
\item add r3, r1, r2
\item ldr r1, [r0, r2]
\item str r0, [r1, r2, lsl \#2]
\end{enumerate}


\end{ExerciseList}


\subsection*{Design Problems}

\begin{ExerciseList}
\Exercise
Run your ARM programs on an ARM emulator such as the QEMU (\url{www.qemu.org}) emulator, or $arm-elf-run$ (available at \url{www.gnuarm.com}).

\end{ExerciseList}



















