\vskip 2cm
\begin{flushright}
Solutions by Saumya Sahay $<$saumyasahay98@gmail.com$>$
\end{flushright}
\section*{Exercises}
\vskip 1cm

\setcounter{Exercise}{0}
\setcounter{Answer}{0}

\section*{Assembly Language Concepts}

\begin{ExerciseList}

\Exercise What is the advantage of the register indirect addressing mode over the memory direct
addressing mode?
\Answer :
\begin{enumerate}
\item The advantages of register indirect addressing mode is that it makes accessing data dynamic rather than static as in the case of direct addressing mode.
ode.
\end{enumerate}

\Exercise 
When is the base-offset addressing mode useful?
\Answer :
Base offset is useful to access fields of a structure or objects.

\Exercise
Consider the base-scaled-offset addressing mode, which directs the hardware to automatically
multiply the offset by 4. When is this addressing mode useful?
\Answer :
Based-scaled offset modeis useful while traversing stacks and complexdata structures.

\Exercise Which addressing modes are preferable in a machine with a large number of registers?
\Answer :
Base index offset, Base index addressing modes,Register indirect addressing.

\Exercise Which addressing modes are preferable in a machine with very few registers?
\Answer :
Immediate addressing,Memory-direct addressing modes.

\Exercise Assume that we are constrained to have at the most two operands per instruction. Design
a format for arithmetic instructions such as add and multiply in this setting.
\Answer :
\begin{enumerate}
\item add R0,R1 (stores the value of R0+R1 in R0)
\item sub R0,R1 (subtracts R1 from R0 and stores in R0)
\end{enumerate}


\end{ExerciseList}

\section*{Assembly Programming}

\begin{ExerciseList}

\Exercise Write simple assembly code snippets in \simplerisc to compute the following: 

\begin{enumerate}[i) ]
\item $a + b + c$
\item $a + b - c/d$
\item $(a + b)*3 - c/d$
\item $a/b - (c*d)/3$
\item $(a\ll2) - (b\gg 3)$ (($\ll$ (left shift logical), $\gg$ (left shift arithmetic))
\end{enumerate}

\Answer :
We store the values of a,b,c and d in registers R0,R1,R2and R3 respectively.\\
\begin{enumerate}
\item \begin{Verbatim}[frame=single]
add R0,R0,R1
add R0,R2,0
\end{Verbatim}
\item \begin{Verbatim}[frame=single]
div R2,R2,R3
add R0,R0,R1
sub R0,R0,R2
\end{Verbatim}
\item \begin{Verbatim}[frame=single]
add R0,R0,R1
mul R0,R0,3
div R2,R2,R3
sub R0,R0,R2
\end{Verbatim}
\item \begin{Verbatim}[frame=single]
div R0,R0,R1
mul R2,R2,R3
div R2,R2,3
sub R0,R0,R2
\end{Verbatim}
\item
\begin{Verbatim}[frame=single]
lsl R0,R0,2
asr R1,R1,3
sub R0,R0,R1
\end{Verbatim}
\end{enumerate}

\Exercise Write a program to load the value $0xFFEDFC00$ into $r0$. Try to minimise
the number of instructions.
\Answer :
\begin{Verbatim}
movh R0,0xFFED
addu R0,R0,0xFC00
\end{Verbatim}

\Exercise
Write an assembly program to set the $5^{th}$ bit 
of register $r0$ to the value of the $3^{rd}$ bit of $r1$. Keep the rest of the contents
of $r0$ the same. The convention
is that the LSB is the first bit, and the MSB 
is the $32^{nd}$ bit. (Use less than or equal to 5 $SimpleRisc$ assembly statements) 
\Answer:
\begin{Verbatim}[frame=single]
and R1,R1,4
lsl R1,R1,2
and R0,R0,0xffef
or R0,R0,R1
\end{Verbatim}

\Exercise
Write a program in \simplerisc assembly to convert an integer stored in memory from the little endian to the big endian format.
\Answer :
\begin{Verbatim}[frame=single]
mov R0,R1
lsl R2,R0,24
lsr R3,R0,24
movu R6,0xff00
movh R7,0x00ff
and R4,R0,R6
lsl R4,R4,8
and R5,R0,R7
lsR R5,R5,8
add R3,R3,R5
add R3,R3,R4
add R2,R2,R3
st R2,[R1]
.print R2
\end{Verbatim}

\Exercise 
Write a program in \simplerisc assembly to compute the factorial of a positive number using an
iterative algorithm.
\Answer:
\begin{Verbatim}[frame=single]
mov R1,1
mov R2,R0
.loop: 
	mul R1,R1,R2
	sub R2,R2,1
	cmp R2,1
	bgt .loop
.exit: .print R1

\end{Verbatim}

\Exercise 
Write a program in \simplerisc assembly to find if a number is prime.
\Answer:
\begin{Verbatim}[frame=single]
mov R2,2
.loop:
	mod R3,R1,R2
	cmp R3,0
	beq .no
	add R2,R2,1
	cmp R2,R1
	bgt .loop
	mov R0,1
	b .exit
.no:
	mov R0,0
.exit:
	.print R0

\end{Verbatim}
\Exercise Write a program in \simplerisc assembly to test if a number is a perfect
square. 
\Answer :
\begin{Verbatim}[frame=single]
mov R2,0
.loop:
	mul R3,R2,R2
	cmp R3,R1
	beq .yes
	add R2,R2,1
	cmp R1,R2
	bgt .loop
	mov R0,0
	b .exit
.yes:
	mov R0,1
.exit:
	.print R0
\end{Verbatim}
\Exercise Given a 32-bit integer in $r3$,
write a \simplerisc assembly program to count the number of 1 to 0 transitions in it. 
\Answer: 
\begin{Verbatim}[frame=single]
movu R1,0x0001
mov R2,32
mov R4,0
mov R5,0
mov R6,R0
.loop1:
	cmp R2,0
	beq .exit
	lsr R6,R0,R5
	and R3,R6,R1
	sub R2,R2,1
	add R5,R5,1
	cmp R3,1
	beq .loop1
.loop2:
	cmp R2,0
	beq .exit
	lsr R6,R0,R5
	and R3,R6,R1
	sub R2,R2,1
	add R5,R5,1
	cmp R3,0
	beq .loop2
	add R4,R4,1
	b .loop1
.exit:
	.print R4
\end{Verbatim}
\Exercise[difficulty=1] 
Write a program in \simplerisc assembly to find the smallest number that is a sum of
two different pairs of cubes. [Note: 1729 is the Hardy-Ramanujan number. $1729 =12^3 + 1^3
= 10^3 + 9^3$].
\Answer :
\begin{Verbatim}[frame=single]

.main:
	mov r0,0
	mov r4,0
	mov r5,0
	mov r6,0
.loop1:
	mov r3,0
	mov r1,1
.loop2:
	mul r4,r1,r1
	mul r4,r4,r1
	cmp r4,r0
	bgt .break2
	mov r2,1
.loop3:
	cmp r2,r1
	bgt .break3
	mul r5,r2,r2
	mul r5,r5,r2
	add r6,r4,r5
	cmp r6,r0
	beq .check
	b .next
.check:
	add r3,r3,1
	cmp r3,2
	beq .break1
.next:
	add r2,r2,1
	b .loop3
.break3:
	add r1,r1,1	
	b .loop2
.break2:
 	add r0,r0,1
   	b .loop1
.break1:
	.print r0

\end{Verbatim}

\Exercise Write a \simplerisc 
assembly program that checks if a 32-bit number is a palindrome.
Assume that the input is available in $r3$. The program should set $r4$
 to 1 if it is a palindrome, otherwise $r4$ should contain a 0.
A palindrome is a number which is the same when read from both
 sides. For example, 1001 is a 4 bit palindrome.
\Answer :
\begin{Verbatim}[frame=single]
	mov R0,1044480
	mov R3,0
	mov R4,31
	movu R2,0x0001
	mov R5,0
	andu R1,R0,0xffff
	.loop:
		and R3,R2,R1
		lsl R2,R2,1
		lsl R3,R3,R4
		add R5,R5,R3
		sub R4,R4,2
		cmp R4,0
		bgt .loop
		lsr R0,R0,16
		lsr R5,R5,16
		sub R0,R5,R0
		cmp R0,0
		beq .palin
		mov R0,0
		b .exit
	.palin:
		mov R0,1
	.exit:
	
\end{Verbatim}
\Exercise Design a \simplerisc program that examines a 32-bit 
value stored in $r1$ and counts the number of contiguous sequences of 1s. For example, the value:
\begin{displaymath}
0111 0001 0001 1110 1100 0111 0001 1111 
\end{displaymath}
contains six sequences of 1s. Write the result in $r2$.
\Answer :
\begin{Verbatim}[frame=single]
movu R1,0x0001
mov R2,32
mov R4,0
mov R5,0
mov R6,R0
.loop1:
	cmp R2,0
	beq .exit
	lsr R6,R0,R5
	and R3,R6,R1
	sub R2,R2,1
	add R5,R5,1
	cmp R3,0
	beq .loop1
	add R4,R4,1
.loop2:
	cmp R2,0
	beq .exit
	lsr R6,R0,R5
	and R3,R6,R1
	sub R2,R2,1
	add R5,R5,1
	cmp R3,1
	beq .loop2
	b .loop1
.exit:
	.print R4

\end{Verbatim}
\Exercise[difficulty=2]
Write a program in \simplerisc assembly to subtract two 64 bit numbers stored
in two registers.

\Exercise[difficulty=2]
In some cases, we can rotate an integer to the right by $n$  positions (less than or equal to 31) so that we obtain the same number.
For example: a 8-bit number 11011011 can be right rotated by 3 or 6
places to obtain the same number.
Write an assembly program
to {\em efficiently} count the number of ways we can rotate a number to the right such that the result is 
equal to the original number. 
\Answer:
\begin{Verbatim}[frame=single]
mov R0,1431655765
mov R1,31
mov R3,R0
.loop:	cmp R1,0
	beq .exit
	lsl R2,R0,31
	lsr R0,R0,1
	sub R1,R1,1
	movu R5,0xffff
	addh R5,R5,0x7fff
	and R0,R5,R0
 	or R0,R0,R2
	cmp R3,R0
 	beq .inc
	 b .loop
.inc :  
	 add R4,R4,1
	 b .loop
.exit : 
	.print R4
\end{Verbatim}
\Exercise[difficulty=2]
A number is known as a cubic Armstrong number if the sum of the cubes of the decimal digits is equal to the
number itself. For example, 153 is a cubic Armstrong number (153 = $1^3 + 5^3 + 3^3$). You are given a number in register, $r0$,
and it is known to be between 1 and 1 million. Can you write a piece of assembly code in $SimpleRisc$ to find out
if this number is an Armstrong number. Save 1 in $r1$ if it is an Armstrong number; otherwise, save 0.
\Answer :
\begin{Verbatim}[frame=single]
mov R4,R1
mov R2,0
.loop:
	mod R2,R1,10
	mul R3,R2,R2
	mul R3,R3,R2
	add R0,R0,R3
	div R1,R1,10
	cmp R1,0
	bgt .loop
	cmp R4,R0
	beq .arms
	mov R5,0
	b .exit
.arms:
	mov R5,1
.exit:
	.print R5
\end{Verbatim}

\Exercise[difficulty=3]
Write a \simplerisc assembly language program to find the greatest common divisor of two binary numbers $u$ and $v$.
Assume the two inputs (positive integers) 
to be available in $r3$ and $r4$. Store the result in $r5$.
[HINT: The gcd of two even numbers $u$ and $v$ is $2*gcd(u/2,v/2)$]
\Answer: 
\begin{Verbatim}[frame=single]
mov R0,42
mov R1,98
mov R4,1
 	cmp R0,0
 	beq .exit
	cmp R1,0 
	beq .case1
.cases:	
	cmp R0,R1
	bgt .exchange
	mod R3,R1,R0
	cmp R3,0
	beq .exit

.case2:
	mod R3,R0,2
	cmp R3,0
	beq .divide1
	b .case3

.divide1:
	div R0,R0,2
	mod R3,R1,2
	cmp R3,0
	beq .check
	b .cases

.check:
	div R1,R1,2
	mul R4,R4,2
	b .cases


.case3:
	mod R3,R1,2
	cmp R3,0
	beq .divide2
	
	sub R0,R1,R0
	div R0,R0,2
	b .cases

.divide2:
	div R1,R1,2
	b .cases

.case1: 
	mov R0,R1
	b .exit

.exchange:
	mov R3,R0
	mov R0,R1
	mov R1,R3
	b .cases
.exit:
	mul R0,R0,R4
\end{Verbatim}

\end{ExerciseList}

\section*{Instruction Set Encoding}

\begin{ExerciseList}

\Exercise Encode the following \simplerisc instructions:

\begin{enumerate}[i) ]
\item $sub$ $sp$, $sp$, 4
\item $mov$ $r4$, $r5$
\item $addu$ $r4$, $r4$, 3
\item $ret$
\item $ld$ $r0$, $[sp]$
\item $st$ $r4$, $8[r9]$
\end{enumerate}
\Answer :
\begin{enumerate}
\item Binary: 00001 1 1110 1110 000000000000000100\\
Hexadecimal : 0x0FB80004  
\item Binary: 01000 0 0100 0000 000000000000000101\\
Hexadecimal:0x41000005
\item Binary: 00000 1 0100 0100 010000000000000011\\
Hexadecimal: 0x05110003
\item Binary: 10100\\
Hexadecimal: 0x00000014
\item Binary:01110 1 0000 1110 000000000000000000\\
Hexadecimal : 0x74380000
\item Binary:01111 1 0100 1001 000000000000001000\\
Hexadecimal : 0x7G240008
\end{enumerate}

\end{ExerciseList}
\section*{Design Problems}
\begin{ExerciseList}
\Exercise

Design an emulator for the \simplerisc ISA. The emulator reads an assembly program line by line, checks each
assembly statement for errors, and executes it. 
Furthermore, define two assembler directives namely $.print$, and $.encode$ to print data on the screen.
The $.print$ directive takes a register or memory location as input. When the emulator encounters the $.print$
directive, it prints the value in the register or memory location to the screen. Similarly, when the emulator
encounters the $.encode$ directive it prints the 32 bit encoding of the instruction on the screen. Additionally,
it needs to also execute the instruction.

\end{ExerciseList}


























