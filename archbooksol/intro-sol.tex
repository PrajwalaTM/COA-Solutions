\vskip 1cm
 \begin{flushright}
Solutions by Saumya Sahay $<$saumyasahay98@gmail.com$>$
\end{flushright}
 
\section*{Exercises}
\vskip 1cm
                                                                                     

\section*{Processor and Instruction Set}

\begin{ExerciseList}
\Exercise
Find out the model and make of at least 5 processors in devices around you. The devices can
include desktops, laptops, cell phones, and tablet PCs. 

\Exercise
Make a list of peripheral I/O devices for computers. Keyboards are mice are common devices.
Search for uncommon devices. (HINT: joysticks, game controllers, fax machines)

\Exercise
What are the four properties of an instruction set?

\Exercise
Design an instruction set for a simple processor that needs to perform the following
operations: 
\begin{enumerate}
\item Add two registers
\item Subtract two registers
\end{enumerate}

\Exercise
Design an instruction set for a simple processor that needs to perform the following
operations: 

\begin{enumerate}
\item Add two registers
\item Save a register to memory
\item Load a register from memory
\item Divide a value in a register by two
\end{enumerate}

\Exercise
Design an instruction set
to perform the basic arithmetic operations -- add, subtract, multiply, and divide. 
Assume that all the instructions can have just one operand.

\Exercise[difficulty=1]
Consider the $sbn$ instruction that subtracts the second operand from the first operand,
and branches to the instruction specified by the label (third operand), if the result is negative.
Write a small program using only the $sbn$ instruction to compute the factorial of a positive
number. 
\Answer:
We assume that the variable one is initialised to 1, index to the number whose factorial is to be found and fact to 1.
fact stores the factorial of the number in index at the end of the execution of the program.

\begin{Verbatim}[frame=single]
1:  sbn temp1,temp1
2:  sbn temp1,index
3:  sbn a,temp1
4:  sbn temp,temp
5:  sbn c,c
6:  sbn temp,fact
7:  sbn a,one,11
8:  sbn c,temp
9:  sbn temp2,temp2
10: sbn temp2,one,7 
11: sbn temp2,temp2
12: sbn temp2,c
13: sbn fact,fact
14: sbn fact,c
15: sbn index,one
16: sbn temp, temp
17: sbn temp,index,1
18:<exit>
\end{Verbatim}
\Exercise[difficulty=1]
Write a small program using only the $sbn$ instruction to test if a number is prime. 

\Answer:
We assume that the variable one is initialised to 1, index to the number whose factorial is to be found and b to 1.
The program jumps to label prime if the number is prime and to notprime if the number is not prime.

\begin{Verbatim}[frame=single]
1: sbn a,a
2: sbn temp,temp
3: sbn temp,num
4: sbn a,num,
5: sbn temp1,temp
6: sbn temp1,one
7: sbn b,temp1
8: sbn temp2,temp2
9: sbn temp2,temp
10: sbn temp2,b
11: sbn temp2,one,prime
12: sbn c,c
13: sbn temp4,temp4
14: sbn quo,quo
15: sbn c,one
16: sbn a,b,1
17: sbn quo,c
18: sbn temp4,1,16
19: sbn a,one,not_prime
\end{Verbatim}

\end{ExerciseList}

\section*{Theoretical Aspects of an ISA*}
\begin{ExerciseList}

\Exercise
Explain the design of a modified Turing machine.

\Exercise
Prove that the $sbn$ instruction is Turing complete.

\Exercise
Prove that a machine with memory load, store, branch, and subtract instructions is Turing complete.

\Exercise[difficulty=2]
Find out other models of universal machines from the internet and compare them with
Turing Machines. 
\end{ExerciseList}

\section*{Practical Machine Models}

\begin{ExerciseList}
\Exercise What is the difference between the Harvard architecture and Von Neumann
architecture?

\Exercise What is a register machine? 

\Exercise What is a stack machine? 


\Exercise
Write a program to compute $\mathbf{a + b + c - d}$ on a stack machine. 
\Answer:
\begin{Verbatim}[frame=single]
push d 
push c 
subtract //c-d
push a 
push b 
add    //a+b
add    //a+b+c-d 
pop w
\end{Verbatim}
\Exercise
Write a program to compute $\mathbf{a + b + (c - d)*3}$  on a stack machine. 
\Answer:
\begin{Verbatim}[frame=single]
push d 
push c 
subtract //c-d
push 3
multiply //(c-d)*3
push a 
push b 
add    //a+b
add    //a+b+(c-d)*3
pop w
\end{Verbatim}
\Exercise
Write a program to compute $\mathbf{(a + b/c) *(c - d)+e}$
on a stack machine. 
\Answer:
\begin{Verbatim}[frame=single]
push c
push b
divide // b/c
push a
add    //(a+(b/c))
push d
push c
subtract
multiply   //(a+(b/c))*(c-d)
push e
add    //(a+(b/c))*(c-d)+e
pop w

\end{Verbatim}
\Exercise[difficulty=2]
Try to search the internet, and find answers to the following questions.  
\begin{enumerate}
\item When is having a separate instruction memory more beneficial?
\item When is having a combined instruction and data memory more beneficial?
\end{enumerate}


\end{ExerciseList}



