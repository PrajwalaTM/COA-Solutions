\vskip 2cm

\section*{Exercises}
\vskip 1cm

\setcounter{Exercise}{0}
\setcounter{Answer}{0}

\subsection*{Addition}

\begin{ExerciseList}

\Exercise
Design a circuit to find the 1's complement of a number using half adders only. 

\Exercise
Design a circuit to find the 2's complement of a number using half adders and logic gates. 

\Exercise
Assume that the latency of a full adder is 2ns, and that of a half adder is 1ns. What is the latency
of a 32 bit ripple carry adder?

\Exercise[difficulty=1]
Design a carry-select adder to add two n-bit numbers in $O(\sqrt{n})$ time, where
the sizes of the blocks are $1, 2, ... ,m$ respectively.

\Exercise
Explain the operation of a carry lookahead adder.

\Exercise[difficulty=1]
Suppose there is an architecture which supports numbers in base 3 instead of base 2. Design a Carry Lookahead Adder for
this system. Assume that you have a simple full-adder which adds numbers in base 3. 

\Answer
The following cases generate a carry:


\begin{enumerate}
	\item $1+2 = 10$
	\item $2+1 = 10$
	\item $2+2 = 11$
\end{enumerate}

$G_i = 1$ if it falls in the above 3 cases, otherwise $G_i = 0$. \\
The following case propagates a carry:
 
\begin{enumerate}
	\item $1+1 = 2$
\end{enumerate}

$P_i = 1$ for the only above case, otherwise $P_i = 0$.\\
The remaining things remain same as before.\\
$P_{ij} = P_{ik}\; .\; P_{kj}$ \hspace{24 mm} $i<k<j$\\
$G_{ij} = G_{kj}\;+\;(P_{kj}\;.\;G_{ik})$\hspace{10 mm} $i<k<j$\\ \\
No changes are required for Ripple Carry Adder and Carry Select Adder.



\Exercise[difficulty=1]
Most of the time, a carry does not propagate till the end.
In such cases, the correct output is available much before the worst case delay.
Modify a ripple carry adder to consider such cases and set an output line to high as soon as
the correct output is available.

\Answer
Defining Cumulative Carry Generate function $CG_i$ for every $i$ in Range 0 to 31:\\
$CG_i=G_{31}\;+\;G_{30}\;+\;......\;+\;G_i$\\
where $G_i$ is the Generate function corresponding to position $i$ in Carry Lookahead Adder.\\ The adder first computes
all these $CG_i$ (in constant amount of time). Then as soon as a carry $C_i$ is generated for position i, the adder
computes $(CG_{i+1}\; and \;C_i)$. If it is zero, it implies that there will be no carry propagation and generation any
further and the output from the adder is correct now. So adder can set an output line high indicating that it has
completed the addition.


\Exercise[difficulty=1]
Design an optimal adder, which uses only the propagate function, and simple
logic operations. It should NOT use the generate function. What is its time and space complexity?

\Answer
Let there be a block (i,j). Carry out of this block $C_{i,j}$:\\
$C_{i,j} = P_{i,j}$ + Carry generated in the block,\\
where $P_{i,j}$ is the propagate function for block (i,j).\\
We will use an alternative approach to get the carry generated in the block. First of all we divide the number in
$\sqrt{n}$ blocks, each of size $\sqrt{n}$. Now we can add the numbers is a block to get the carry out. This will take
O($\sqrt{n}$) amount of time. Calculating propagate functions will take O($\log n$) time. Now that we have the carry out
for each block, we can add the two complete numbers. Total time taken to add is O($\sqrt{n}$).

\Exercise
Design a hardware structure to compute the sum of $m$, $n$ bit numbers. Make it run as fast as possible. Show the design
of the structure. Compute a tight bound on its asymptotic time complexity. [NOTE: Computing the time
complexity is not as simple as it seems].


\Exercise[difficulty=2]
You are given a probabilistic adder, which adds two numbers and yields the output
ensuring that each bit is correct with probability, $a$.
In other words, a bit in the output may be wrong with probability, $(1-a)$, and this event is independent
of other bits being incorrect.
How will you add two numbers using using probabilistic adders 
ensuring that each output bit is correct with at least a probability of $b$, where $b>a$?


\Answer
A simple idea is to use multiple such adders in parallel. The bit at position $i$ of the final output should be the
result of majority function of the bits at position $i$ of the outputs produced by each adder. Suppose we use $n$ adders
in parallel (where $n$ is odd), then we have to find $n$ such that probability that at least $(n+1)/2$ bits are correct
$\geq b$. In other words, we have to find $n$ such that \\ \\ $\sum\limits_{r=(n+1)/2}^{n} \ a^{r}.(1-a)^{n-r} \geq b$
\\ \\
Solving the above equation, given the values of $a$ and $b$, we will have $n\geq k$, where $k$ is some constant. The
optimal number of adders is $ceil(k)$.

\Exercise[difficulty=3]
How frequently does the carry propagate to the end for most numbers? Answer: Very infrequently.
In most cases, the carry does not propagate beyond a couple of bits. Let us 
design an approximately correct adder. 
The insight is that a carry does not propagate by more than $k$ positions 
most of the time. Formally, we have: \\
{\bf Assumption 1:} While adding two numbers, the largest length of a chain of propagates
is at most $k$. \\

Design an optimal adder
in this case that has time complexity $O(log\, k)$ assuming that
Assumption 1 holds all the time. Now design a circuit to check if assumption 1
has ever been violated. Verma et. al.~\cite{brisk} proved that $k$ is equal to $O(log(n))$
with very high probability. Voila, we have an exactly correct adder,
which runs most of
the time in $O(log(log(n)))$ time.!!! \\


\Answer {\bf Circuit 1:}\\
Find out the generate function for each bit. Wherever generate function $G_i = 1$, select a chunk of $k$ bits starting
from there and compute the propagate functions for that chunk. Calculating generate function for all bits will take
$O(1)$ time and subsequent computation for propagate function will take $O(\log k)$ time. The complete structure is
similar to Carry Lookahead Adder except we do not need to compute generate and propagate functions for chunks larger
than k bits in size. Addition is done in the similar way.\\ {\bf Circuit 2:}\\
This circuit will check the carry-out from the chunks of k bits which were selected by circuit 1. If the carry-out from
each chunk is equal to 0, then there is no error. If the carry-out from any of the chunk becomes 1, then this circuit
reports an error and the addition is performed using different technique or by different adder.  

\Exercise[difficulty=3]
Let us consider two n bit binary numbers, $A$, and $B$. Further
assume that the probability of a bit being equal to 1 is $p$ in $A$, and $q$ in $B$. Let
us consider ($A + B$) as one large chunk(block).

(a) What are the expected values of generate and propagate functions of this
block as $n$ tends to $\infty$?

(b) If $p = q = \frac{1}{2}$,
 what are the values of these functions?
(c) What can we infer from the answer to part (b) regarding the fundamental
limits of binary addition?
\end{ExerciseList}

\section*{Multiplication}

\begin{ExerciseList}

\Exercise
\label{ex:mulexec}
Write a program  in assembly language (any variant) to multiply two unsigned 32-bit numbers given in
registers $r0$ and $r1$ 
and store the product in registers $r2$ (LSB) and $r3$ (MSB). Instead of using the multiply
instruction, simulate the algorithm
of the iterative multiplier. 

\Answer Code:
\begin{Verbatim}[frame=single]
L1:
	and r5, r3, #1
	cmp r5, #1
	addeq r2, r2, r0
	and r5, r2, #1
	mov r2, r2, lsr #1
	mov r3, r3, lsr #1
	cmp r5, #1
	addeq r3, r3, #0x80000000
	add r4, r4, #1
	cmp r4, #32
	bne L1
	mov pc, lr
multiply:
	mov r2, #0
	mov r3, r1
	mov r4, #0
	bl L1
\end{Verbatim}


\Exercise
Extend the solution to Exercise~\ref{ex:mulexec} for 32 bit signed integers.

\Answer
If a number is negative, take 2's compliment of it to make it positive. Do usual unsigned multiplication for the two
31-bit positive numbers to get the positive product. Calculate the sign of the product from the sign of the initial
multiplicand and initial multiplier : product is negative if and only if their signs are different. If the final product
is positive, leave it as it is, otherwise, take 2's compliment of the product.

\Exercise[difficulty=1]
Normally, in the Booth's algorithm, we consider the current bit, and the previous bit. Based on these two values, we
decide whether we need to add or subtract a shifted version of the multiplicand. This is known as the radix-2 Booth's
algorithm, because we are considering two bits at one time. There is a variation of Booth's algorithm, called radix-4
Booth's algorithm in which we consider 3 bits at a time. Show how this algorithm is faster than the original radix-2
Booth's algorithm. How will you implement this algorithm ?

\Answer
In this algorithm, we consider 3 bits at a time in a block, each block overlaps the previous block by 1 bit. We start
making groups from LSB and proceed towards MSB. Thus, by doing this, we essentially halve the number of groups formed;
since each alternate bit is now covered by only 1 group, while each bit (except for LSB and MSB) previously (in radix-2)
was covered by 2 groups. Following are the possibilities of the bit pattern in a group and corresponding product P.
Assume that the MSB of the group is at position i. Let M be the multiplicand.\\ 

\begin{tabular}{|l|c|}
\hline
Block & $P$ \\
\hline \hline
000 &	$P$ \\
\hline
001	& $P + 2^{i-1} * M$ \\
\hline
010	& $P + 2^{i-1} * M$ \\
\hline
011	& $P + 2^{i} * M$ \\
\hline
100	& $P - 2^{i} * M$ \\
\hline
101	& $P -2^{i-1} * M$\\
\hline
110	& $P -2^{i-1} * M$ \\
\hline
111	& $P$ \\
\hline
\end{tabular} \\

** Take care of boundary cases of LSB and MSB

\Exercise[difficulty=1]
Assume that in the sizes of the $U$ and $V$ registers are 32 bits in a 32 bit Booth multiplier. Is it possible to have an
overflow? Answer the question with an example. [HINT: Can we have an overflow in the first iteration itself?]

\Exercise[difficulty=1]
Prove the correctness of the Booth multiplier in your own words.

\Exercise
Explain the design of the Wallace tree multiplier. What is its asymptotic time complexity.

\Exercise[difficulty=2] Design a Wallace tree multiplier to multiply two signed 32 bit numbers, and save the result in a 32 bit
register. How do we detect overflows in this case?


\end{ExerciseList}

\section*{Division}

\begin{ExerciseList}

\Exercise Implementation of division using an assembly program.
\begin{enumerate}
\item[i) ] Write an assembly program for restoring division.
\item[ii) ] Write an assembly program for non-restoring division.
\end{enumerate}

\Answer Restoring division:
\begin{Verbatim}[frame=single]
L1:
	and r3, r8, #0x80000000
	mov r8, r8, lsl #1
	mov r7, r7, lsl #1
	cmp r3, #0
	addne r7, r7, #1
	cmp r7, r1
	subge r7, r7, r1
	addge r8, r8, #1
	add r4, r4, #1
	cmp r4, #32
	bne L1
	mov pc, lr
divideRestore:
	mov r8, r0
	mov r4, #0
	bl L1
\end{Verbatim}


Non-restoring division:
\begin{Verbatim}[frame=single]
L1:
	and r3, r8, #0x80000000
	mov r8, r8, lsl #1
	mov r7, r7, lsl #1
	cmp r3, #0
	addne r7, r7, #1
	cmp r7, #0
	subge r7, r7, r1
	addlt r7, r7, r1
	cmp r7, #0
	addge r8, r8, #1
	add r4, r4, #1
	cmp r4, #32
	bne L1
	cmp r7, #0
	addlt r7, r7, r1
	mov pc, lr
divideNonRestore:
	mov r8, r0
	mov r4, #0
	bl L1
\end{Verbatim}

\Exercise[difficulty=1]
Design an $O(log(n)^k)$ time algorithm to find out if a number is divisible by 3. Try to minimise $k$. 

\Answer
If a number is divisible by 3, the absolute value of difference of the sum of bits at odd position and even positions is 
divisible by 3. \\
That is:\\
A number n is divisible by 3 iff:\\
$(\sum b_{2k+1} - \sum b_{2k})$ MOD $3 =0$\\
where $b_i$ is a bit of n at position i.\\
or,\\
$((b_1-b_0) + (b_3-b_2) + (b_5-b_4) + (b_7-b_6) .... )$ MOD 3 $= 0$\\
$\Rightarrow (((b_1-b_0)$ MOD 3 $+ (b_3-b_2)$ MOD 3$)$ MOD 3 $+ ((b_5-b_4)$ MOD 3 $+ (b_7-b_6)$ MOD 3$)$ MOD 3 $.... )$ MOD 3 $= 0 $\\ ...and so on.
So essentially, we build a circuit which calculates $(b_{2k+1} - b_{2k})$ MOD 3 in the first step, and then combines the
result by adding the subsequent pairs modulo 3, until a final result is obtained. If this final result in 0, the number
is divisible by 3.\\
The following figure shows an example of an 8 bit number $(11100111)_2$ through our circuit:\\

\begin{center}
%\includegraphics[bb=0 0 200 350,scale=0.5]{div3.png}
\end{center}

Total number of levels in the circuit = $log_2(n)$\\
Therefore, $k=1$;

\Exercise[difficulty=1]
Design an $O(log(n)^k)$ time algorithm to find out if a number is divisible by 5. Try to minimise $k$. 

\Exercise[difficulty=2]
Design a fast algorithm to compute the remainder of the division of an unsigned number
by a number of the form ($2^m + 1$). What is its asymptotic time complexity?

\Exercise[difficulty=2]
Design a fast algorithm to compute the remainder of the division of an unsigned number
by a number of the form ($2^m - 1$). What is its asymptotic time complexity?

\Exercise[difficulty=2]
Design an $O((log(uv))^2)$ algorithm to find the greatest common divisor of two binary numbers $u$ and $v$.
[HINT: The gcd of two even numbers $u$ and $v$ is $2*gcd(u/2,v/2)$]
\Answer
The algorithm, also known as Stein's algorithm is as follows:

\begin{enumerate}
	\item[1. ] $gcd (v, v) = v$
	\item[2. ] $gcd(0, v) = v$, and $gcd(u, 0) = u$ 
	\item[3. ] If $u$ and $v$ are both even, then $gcd(u, v) = 2\:.\:gcd(u/2, v/2)$
	\item[4. ] If $u$ is even and $v$ is odd, then $gcd(u, v) = gcd(u/2, v)$, because $2$ is not a common divisor. Similarly for the other case, then $gcd(u, v) = gcd(u, v/2)$.
	\item[5. ] If $u$ and $v$ are both odd:
	\begin{enumerate}
		\item[i) ] If $u \geq v$, then $gcd(u, v) = gcd((u - v)/2, v)$ 
		\item[ii) ] If $u < v$, then $gcd(u, v) = gcd((v - u)/2, u)$
	\end{enumerate}
\end{enumerate}
Each step reduces at least one of the operand by at least a factor of two.  Thus, this algorithm runs in
$O((log(uv))^2)$ in worst case. Or in other words, this algorithm takes time proportional to square of number of bits in
$u$ and $v$ together. 


\end{ExerciseList}

\section*{Floating Point Arithmetic}

\begin{ExerciseList}

\Exercise
Give the simplest possible algorithm to compare two 32 bit IEEE 754 floating point numbers. Do not consider $\pm
\infty$, NAN, and (negative 0).  Prove that your algorithm is correct. What is its time complexity ?

\Answer
Just compare the numbers like integers. IEEE 754 floating point numbers have their 1st bit as sign bit (which is same as
integers), next 8 bits store the exponent and the last 23 bits store mantissa. When we compare two fractions, we first
check the sign, then the exponent and finally the mantissa. If the signs are different, the comparison is done and the
positive one is greater (which is done correctly by comparing them as normal integers). If the two fractions are
positive, the one with greater exponent will be greater. This again will be done correctly by normal integer comparison
as exponent is placed to the left of mantissa. If the two fractions are negative, the one with greater exponent is
smaller, again done correctly by integer comparison: negative integer with higher first 8 bits after MSB is smaller.
Then comes the mantissa which is also checked correctly by integer comparison. \\ Time complexity = O(log n) taken for
subtraction and sign comparison, where n is the number of bits.



\Exercise 
Design a circuit to compute $\lceil log_2(n) \rceil$. What is its asymptotic time complexity?

\Exercise $A$ and $B$, are saved in the computer as $A'$ and $B'$. Neglecting any further truncation or roundoff errors, show that the relative error of the product is approximately the sum of the relative errors of the factors.  \Answer Let $\alpha$ be the relative error in $A'$ and $\beta$ be the relative error in $B'$. \\ So, $|\alpha| < 1$ and $|\beta| < 1$ \\ $A=A' * (1+\alpha) \\ B=B' * (1+\beta) \\ A*B = A'B' * ( 1+ \alpha + \beta + \alpha\beta) \\ = A'B' * (1+\alpha+\beta)(1+\alpha\beta/(1+\alpha+\beta)) \\ $
Since $\alpha$ and $\beta$ are small quantities with absolute value less than 1, $\alpha\beta / (1+\alpha+\beta) \longrightarrow 0$\\
Therefore, \\
$A*B \approx A'B' * (1+ (\alpha+\beta))\\
		= A'B' *(1+\delta) $\\
where $\delta = \alpha + \beta$

\Exercise
Show the flowchart for floating point addition.

\Exercise
Show the flowchart for floating point multiplication.

\Exercise
Can we use regular floating point division for dividing integers also? If not, then how can we modify the algorithm
for performing integer division? 

\Exercise
Describe in detail how the ``round to nearest'' rounding mode is implemented.

\Exercise[difficulty=3]
We wish to compute the square root of a floating point number in hardware using the Newton-Raphson method.
Outline the details of an algorithm, prove it, and compute its computational complexity.
Follow the following sequence of steps. 

\begin{enumerate}
\item Find an appropriate objective function. 
\item Find the equation of the tangent, and the point at which it intersects the x-axis. 
\item Find an error function. 
\item Calculate an appropriate initial guess for $x$. 
\item Prove that the magnitude of the error is less than 1.
\item Prove that the error decreases at least by a constant factor per iteration. 
\item Evaluate the asymptotic complexity of the algorithm. 
\end{enumerate} 

\Answer 
\begin{enumerate}
\item Objective function: $f(x) = x^2 - b$
Assume that $1/2 \le b < 2$. Any floating point number in the normal form can be brought to this form by
rounding the exponent to the nearest even number. Example. $1.5 \times 2^{-3} = 0.75 \times 2^{-2}$. 
\item Slope = 2x. Point at which the tangent intersects the x-axis ($x_2$):
\begin{equation}
\begin{split}
\frac{x_1^2 - b}{x_1 - x_2} = 2x_1 \\
\Rightarrow x_2 = \frac{x_1^2 + b}{2x_1} 
\end{split}
\end{equation}
\item Error function:  $E(x)  = x - \sqrt{b}$
\begin{equation}
\begin{split}
E(x_2) & = x_2 - \sqrt{b} \\
       & = \frac{x_1^2 + b}{2x_1} - \sqrt{b} \\
       & = \frac{x_1^2 + b - 2x_1\sqrt{b}}{2x_1} \\
       & = \frac{(x_1 - \sqrt{b})^2}{2x_1} \\
       & = \frac{E(x_1)^2}{2x_1}
\end{split}
\end{equation}
\item Initial guess for $x$: x = 1

\item Initial value of the error: $1 - \sqrt{b}$. 
\[
(1/2 \le b < 2) \Rightarrow (1 - \sqrt{2}) < (1 - \sqrt{b}) \le (1 - 1/\sqrt{2}) 
\]
Hence, $\mid 1 - \sqrt{b} \mid < 1$
\item 
Let us prove by induction that $x_n \ge 1/2$, and $b \ge x_n / 2$. 
This is true for the base case $x_1 = 1$. 

Assume the hypothesis is true for $x_n = x'$.
Let us try to prove that the hypothesis holds for $x_{n+1} = x''$.

\begin{equation}
\begin{split}
x'' & = \frac{x'^2 + b}{2x'} \\
   & = \frac{x'}{2} + \frac{b}{2x'} 
\end{split}
\end{equation}

Now, $x'/2 \ge 1/4$ (because, $x' \ge 1/2$), and $b/2x' \ge 1/4$ (because $b \ge x' / 2$).
Hence, $x'' \ge 1/2$.

Let us now prove that $b \ge x''/2$. We have:

\begin{equation}
\begin{split}
 & b \ge x''/2  \\
\Leftrightarrow & b \ge \frac{x'^2 + b}{4x'} \\
\Leftrightarrow & b \ge \frac{x'}{4} + \frac{b}{4x'} \\
\end{split}
\end{equation} 

By the induction hypothesis, $b/2 \ge x'/4$. 
If we prove that $b/2 \ge b/4x'$, we are done.

\begin{equation}
\begin{split}
& \frac{b}{2} \ge  \frac{b}{4x'} \\
\Leftrightarrow & 1 \ge \frac{1}{2x'} \\
\Leftrightarrow & x' \ge \frac{1}{2} (TRUE) \\ 
\end{split}
\end{equation}


We have thus proved the induction, and we can conclude that $x_n \ge 1/2$ for all values of $n$. 
Hence, $\forall n,2x_n \ge 1$.
We thus have: 
\[
E(x_{n+1}) \le E(x_n)^2 
\]

\item We have $log(n)$ steps because, we assume that the precision is till $n$ bits. In each step, 
we have to do a left shift and round (log(n)), addition (log(n)), and division ($log(n)^2$).

Thus, the total time complexity is \fbox{$O(log(n)^3)$}
\end{enumerate}

\noindent {\bf Note to TAs:} : (a and b) should have a reduced form of $b$  (1 mark) (c) 1 mark
(d and e) 1 mark (f) 4 marks (g) 1 mark. Take a look at part (f) carefully, it is the crux of the algorithm.
The proof should be crystal clear (for full marks). If there is any ambiguity, the benefit of doubt goes to the TA.
For this answer, we need to assume that the time complexity of a division is $O(log(n)^2)$ (This is a research
question). 

\end{ExerciseList}

\section*{Design Problems}
\begin{ExerciseList}

\Exercise
Implement an adder, multiplier, in a hardware description language such as VHDL or Verilog.

\Exercise
Extend your design for implementing floating point addition and multiplication.

\Exercise
Read about the SRT division algorithm, comment on its computational complexity, and try to implement it in
VHDL/Verilog.
\end{ExerciseList}










































