\section*{Exercises}
\vskip 1cm

\setcounter{Exercise}{0}
\setcounter{Answer}{0}

\section*{Overview of the I/O System}

\begin{ExerciseList}
\Exercise
What are the roles of the North Bridge and South Bridge chips?
\Answer
The North Bridge Chip is connected to the DRAM memory modules, graphics card and the South Bridge Chip.\\
North Bridge Chip is connected directly to the CPU via the frontside bus (FSB) and thus responsible for tasks that require the highest performance.\\
The South Bridge Chip is connected to all the USB devices including the keyboard and mouse,audio devices,network cards and the hard disk.\\
The South Bridge typically implements the slower capabilities of the motherboard in a North Bridge/South Bridge chipset computer architecture. \\
In systems where they are included, these two chips manage communications between the CPU and other parts of the motherboard, and constitute the core logic chipset of the PC motherboard.
\\
\Exercise
What is the role of the chipset in a motherboard?
\Answer
Chipset: These are a set of chips that are required by the main CPU to connect to the main memory,the I/O devices and to perform system management functions.
Since most processors are connected only to a single bus,or at most 2 or 3 buses, we need to use ancillary chips that connect the processor to a host of different I/O buses. Hence, the Chipset is used to aggregate the traffic from I/O devices,and properly route data generated by the CPU to the correct I/O devices and vice versa. 
\Exercise
Describe the four layers in the I/O system.
\Answer
The functionality of the I/O system can be broadly divided into four different layers: \\
1.Physical Layer :
The physical layer of a bus primarily defines he electrical specifications of the bus. It is divided into two sub layers namely the transmission sublayer and synchronisation sublayer.
2.Data Link Layer:
The data link layer is primarily designed to process logical bits that are read by the physical layer. This layer groups sets of bits into frames , performs error checking,controls the access to the bus and helps implement I/O transactions.
3.Network Layer:
This is layer is primarily concerned with the successful transmission of a set of frames from the processor to an I/O device or vice versa through various chips in the chipset.
4.Protocol Layer:
The topmost layer is referred to as the protocol layer. It is concerned with executing I/O requests end to end. This includes methods for high level communication between the processor and the I/O devices in terms of the message semantics.
\Exercise
Why is it a good idea to design a complex system as a sequence of layers?
\Answer
Various tasks get simplified if we use a complex system design. 
\end{ExerciseList}

\section*{Physical Layer}

\begin{ExerciseList}
\Exercise
What is the advantage of LVDS signalling?
\Answer
Single-ended signalling methods are extremely slow and outdated methods.
LDVS uses two wires to transmit a single signal. The difference between the voltages of these wires is monitored. 
The value transferred is inferred from the sign of the voltage difference. Hence, LVDS operates at low power and can run at very high speeds. Added advantage of LVDS is that our signal will not be affected by the outside noise because LVDS is making use of two wire to transmit a single signal.
Since LVDS is a physical layer specification only, many data communication standards and applications use it.

\Exercise
Draw the circuit diagram of a LVDS transmitter and receiver.

\Exercise
Assume that we are transmitting the bit sequence: 01101110001101.
Show the voltage on the bus as a function of time for the following protocols:
RZ, NRZ, Manchester, NRZI. 

\Exercise
What is the advantage of the NRZI protocol over the NRZ protocol?
\Answer
With NRZ, the presence of a 1 or 0 is marked by a +V or -V signal (never a zero).  While simple to implement, long sequences of ones might look like no signal at all.  The lack of regular signal transitions makes clock recovery from the signal transitions difficult.  Worse still, a long sequence with infrequent changes in voltage causes the DC value (average signal) to drift.  Since that average voltage is used to discriminate between +V and -V, error may be introduced.

NRZI marks a 1 by having a transition from +V to -V (or vice versa) and marks a 0 by having NO transition.  Now, a long string of ones  will recover clock and will  no problem to detect. The average signal problem also goes away since detecting the presence or absence of a transition is easier than trying to detect if the voltage is close enough to a threshold of +V or -V to correctly decode the bit.

\Exercise
Draw the circuit diagram of the receiver of a plesiochronous bus.

\Exercise
What are the advantages of a source synchronous bus?
\Answer
While the problem with the simple synchronous bus and the mesochronous bus is that there we consider the case of a synchronous system where the sender and the receiver share the same clock and it takes a fraction of a cycle to transfer data from the sender to the receiver.
Source synchronous Bus is advantageous because it considers a more realistic scenario where in the clocks of the sender and the receiver might not be the same and the minimal clock drift may become a couple cycles of drift over millions  of cycles.
Source synchronous Bus is also easier to manufacture as compared to the plesiochronous bus. 
\Exercise
Why is it necessary to avoid transitions in the keep out region?

\Exercise
Differentiate between a 2-phase handshake, and a 4-phase handshake?
\Answer
In the 4 phase handshaking, the sender places the data on the bus and sets the strobe. The receiver begins to read data off the bus as soon as it observes the strobe to be set. After it has read the data, it sets the acknowledge line to 1. After the transmitter observes the ack line set to 1, it can be sure of the fact that the receiver has read the data. This approach is more relevant for the RZ and Manchester coding approaches because the transmitter needs to return the default state before transmitting a new bit.\\
On the other hand, in the 2 phase handshaking, after placing the data on the bus, the transmitter toggles the value of the strobe. Subsequently after reading the data, the receiver toggles the value of the ack line.After the transmitter detects that the ack line has been toggled, it starts transmitting the next bit.After a short bit it toggles the value of the strobe, to indicate the presence of data. Again,after reading the bit,the receiver toggles the ack line and the protocol thus continues.
The toggling reduces the number of events that we need to track on the bus. However, this requires us to keep some additional state at the site of the sender and the receiver. 
\Exercise
Why do we set the strobe after the data is stable on the bus?
\Answer
We set the strobe signal to indicate the availability of data. It is important for the data to be stable on the bus because after the strobe is set the receiver immediately starts to read the data values.

\Exercise[difficulty=2]
Design the circuit for a tunable delay element.
\end{ExerciseList}

\section*{Data Link, Network, and Protocol Layer}

\begin{ExerciseList}

\Exercise
What are the different methods for demarcating frames?
\Answer 
Different methods for demarcating frames are:\\
1.Demarcating by Inserting Long pauses\\
2.Bit Count\\
3.Bit/Byte Stuffling\\
\Exercise
Consider a 8-4 SECDED code. Encode the message: 10011001.
\Answer
To encode the message,we need to add appropriate parity bits:\\
P1=1\\
P2=0\\
P3=0\\
P4=0\\
P5=0\\
Hence, the encode message becomes :
1010001010010\\
\Exercise[difficulty=2]
Construct a code that can detect 3 bit errors.

\Exercise[difficulty=2]
Construct a fully distributed arbiter. It should not have any central node that schedules requests.

\Exercise
What is the advantage of split transaction buses?
The advantages of the split transaction buses are that it is simple and portable. All our transfers are essentially unidirectional. We send a message,and then we donot wait for its reply by locking the bus. The sender proceeds with other messages.Whenever,the receiver is ready with the response,it sends a separate message.\\
Along with simplicity,this method also allows us to connect a variety of receivers to the bus. We just need to need to define a simple message semantics,and any receiever circuit that conforms with the semantics can be connected to the bus.
\Exercise
How do we access I/O ports?

\Exercise
What is the benefit of memory mapped I/O?
\Answer
The main advantage of memory mapped I/O is that it uses regular load and store instructions to access I/O devices instead of dedicated I/O instructions.\\
Secondly, the programmer need not be aware of the actual addresses of the I/O ports in the I/O address space. Since dedicated modules in the operating systems,and the memory system,set up a mapping between the I/O address space and the process's virtual address space,the program can be completely oblivious of the semantics of the addressing the I/O ports.
\Exercise
What are the various methods of communication between an I/O device and the processor? 
Order them in the increasing order of processor utilisation.

\Answer:
\begin{enumerate}[(a) ]
  \item Direct Memory Access (DMA)
  \item I/O interrupts
  \item Polling
\end{enumerate}

\Exercise
Assume that for a single polling operation, a processor running at 1 MHz takes 200 cycles. 
A processor polls a printer 1000 times per minute. What percentage of time does the processor
spend in polling?

\Exercise
When is polling more preferable than interrupts?
\Answer

\Exercise
When are interrupts more preferable than polling?
\Answer
The interrupts are preferred over polling because the "disabled" effect of using interrupts is much lower, since you don't have to schedule your I/O threads to continuously poll for more data while there is none available. It has the added benefit that I/O is handled immediately, so if a user types a key, there won't be a pause before the letter appears on the screen while the I/O thread is being scheduled. Combining these issues is a mess - inserting arbitrary stalls into your I/O thread makes polling less resource-intense at the expense of even slower response times. 
\Exercise
Explain the operation of the DMA controller.
\Answer
When the device driver program,determines that there is a necessity to transfer a large amount of data between the memory and an I/O device. Subsequently,it sends the details of the memory region(range of bytes) and the details of the I/O device to the DMA engine. The DMA controller begins the process of transferring data between the main memory and the I/O devices. Once the process is over it sends and interrupt to the processor indicating that the transfer is over.

\end{ExerciseList}


\section*{Hard Disks}

\begin{ExerciseList} 

\Exercise
What is the advantage of zoned recording?
\Answer 
In zoned recording, since we have multiple zones the storage space wasted is not as high as the design with sinngle zone. Secondly the number of zones is typically not very large,the motor of the spindle does not need to readjust its speed frequently.In fact because of spatial locality, the chances of staying within the same zone are fairly high.

\Exercise
Describe the operation of a hard disk?
\Answer
Hard Disks are used to save persistent state in personal computers,servers,and enterprise class systems.
\Exercise
We have a hard disk with the following parameters: 

\vskip 5mm

\begin{center}
\begin{tabular}{|l|l|}
\hline
Seek Time & 50 ms \\
\hline
Rotational Speed & 600 RPM \\
\hline
Bandwidth & 100 MB/s \\
\hline

\end{tabular} 
\end{center}

\vskip 5mm

\begin{enumerate}[(a) ]
\item How long will it take to read 25 MB on an average if 25 MB can be read in one pass. 
\item Assume that we can only read 5 MB in one pass. Then, we need to wait for the platter to rotate by 360$^\circ$
such that the same sector comes under the head again. Now, we can read the next chunk of 5 MB. In this case,
how long will it take to read the entire 25MB chunk? 
\end{enumerate}

\Answer
\begin{enumerate}[(a) ]
\item Seek time = 50 ms\\
Maximum Rotational latency (for one complete rotation) = 100 ms \\
Time to transfer data = 250 ms\\
Total time (max) = 400 ms

\item	Seek time = 50 ms \\
\emph{Assuming that 25 MB of data is transferred only after it has been read completely}\\
Rotational latency (5 rotations) = 500 ms \\
Time to transfer data = 250 ms \\
Total time = 800 ms
\end{enumerate}

\Exercise
Typically, in hard disks, all the heads do not read data in parallel. Why is this the case?

\Exercise
\label{que:cylinder}
Let us assume that we need to read or write long sequences of data. What is the best way of arranging
the sectors on a hard disk? Assume that we ideally do not want to change tracks, and all the tracks in a cylinder
are aligned.

\Exercise[difficulty=1]
Now, let us change the assumptions of Exercise~\ref{que:cylinder}. Assume that it is faster to move to the next track
on the same recording surface, than starting to read from another track in the same cylinder. [NOTE: The question is
not as easy as it sounds.]


\Exercise
Explain the operation of RAID 0,1,2,3,4,5, and 6.

\Answer:\\
{\bf RAID 0}:\\
No redundancy done. Data spread over multiple disks without parity information.\\
{\bf RAID 1}:\\
An exact copy (or mirror) of a set of data on two disks. Write operations done simultaneously on both the disks.\\
{\bf RAID 2}:\\
Stripes done at bit level, with Hamming code for error correction along with parity check.\\
{\bf RAID 3}:\\
Byte level striping with dedicated disk for storing parity. Parity is calculated after reading from every data disk.\\
{\bf RAID 4}:\\
Block level striping with dedicated disk for storing parity. Modified parity is calculated after reading from the modified disk and and the old parity value. \\
{\bf RAID 5}:\\
Similar to RAID 4 but with distributed parity information in spread throughout all the disks, with no single dedicated parity disk.\\
{\bf RAID 6}:\\
Similar to RAID 5 but with two parity blocks distributed across all the disks. It allows recovery from second failure.

 
\Exercise
 
Consider 4 disks D0, D1, D2, D3 and a parity disk P using RAID 4. The following table shows the contents of the disks
for a given sector address, $S$, which is the same across all the disks.

\vskip 4mm

\begin{center}
\begin{tabular}{|c|c|c|c|}
 \hline
 D0 & D1 & D2 & D3 \\
 \hline
0xFF00 & 0x3421 & 0x32FF & 0x98AB \\
 \hline
\end{tabular}
\end{center}

\vskip 4mm

Compute the value of the parity block?
Now the contents of D0 are changed to 0xABD1.
What is the new value of the parity block? 

\Exercise
Assume that we want to read a block as fast as possible, and there are no parallel accesses. Which RAID technology
should we choose?

\Exercise
What is the advantage of RAID 5 over RAID 4? 
\Answer
RAID 5 mitigates the shortcomings of RAID 4. It distributes the parity blocks across all the disks for different rows. For instance the 5th disk stores the parity for the first row and then the 1st disk stores the parity for the second row and the pattern thus continues in a round robin fashion. This ensures that no disk becomes a point of contention and the parity blocks are evenly distributed across all the disks.
\end{ExerciseList}

\section*{Optical and Flash Drives}
\begin{ExerciseList}
\Exercise
What is the main difference between a CD and DVD?
\Answer
A CD belongs to the 1st Generation where as the DVD belongs to the 2nd Generation. Successive generations are typically faster and provide more storage capacity. The capacity of a CD is around 700 MB and that of a DVD is 4.7 GB.

\Exercise
How do we use the NRZI encoding in optical drives?

\Exercise
What is the advantage of running a drive at constant linear velocity over running it at constant angular velocity?

\Exercise[difficulty=1]
For an optical drive that runs at constant linear velocity, what is the relationship between the angular velocity,
and the position of the head?

\Exercise
What are the basic differences between NAND and NOR flash?
\Answer
The NAND and NOR flash cells use different topology.\\
The NOR flash is very similar to a DRAM cell. Its array based layout allows us to access each individual location in the array uniquely.\\
The NAND flash consists of a set of NMOS floating gate transistor in series similar to serires connections in CMOS NAND gates.

\Exercise
Explain {\em wear levelling}, and {\em read disturbance}? How are these issues typically handled in modern Flash devices?
\Answer
A technique to ensure that no single block wears out faster than other blocks is known as wear levelling. Most flash drives implement wear levelling by swapping the contents of a block that is frequently accessed.
Another reliablity issue in flash memories is known as read disturbance. If we read the contents of one page continuously, then the neighbouring transistors in each NAND cell start getting programmed. After thousands of read accesses to just one transistor,the neighbouring transistors start accumulating negative charge in their floatinggates and ultimately get programmed to store a 0 bit. 
\end{ExerciseList}

\section*{Design Problems}

\begin{ExerciseList}

\Exercise
Read more about the Hypertransport\regtm, Intel Quickpath, Infiniband\regtm, and Myrinet\regp protocols? Try to divide their
functionality into layers as we have presented in this chapter.

\end{ExerciseList}
